\documentclass[14pt]{beamer}
\usetheme{naive}
\renewcommand{\baselinestretch}{1.25}

\title{\Naive}
\subtitle{Yet another simple beamer theme}
\author{Takahiro Ueda}
\institute{\url{https://github.com/tueda/beamertheme-naive}}
\date{24 January 2021}

% Well, look up the word "naiive" in the Urban Dictionary :-)
\newcommand{\Naive}{\texorpdfstring{\textsc{Na}ii{ve}}{Naive}}

\begin{document}

\maketitle

\section{Introduction}

\begin{frame}[fragile]{Simple beamer theme}
  There are many beamer theme with simple and clean design.
  Here is another one.

  \pause

  This demo uses:
  \begin{verbatim}
    \documentclass[14pt]{beamer}
    \usetheme{naive}
    \renewcommand{\baselinestretch}{1.25}\end{verbatim}
\end{frame}

\ChangeColorTheme{dark}

\begin{frame}{Coloring (1)}
  There are two pre-defined color themes \textbf{light} and \textbf{dark},
  which can be easily switched by
  \texttt{\textbackslash ChangeColorTheme\{\textit{theme-name}\}}
  between slides.

  \pause

  For more fine tuning of coloring, another command
  \texttt{\textbackslash ChangeColorScheme\{\textit{parameters}\}}
  is available.
\end{frame}

\ChangeColorScheme{%
  fg=red!30!pink,%
  bg=green!10!black%
}

\begin{frame}[fragile]{Coloring (2)}
  For example,
  \begin{verbatim}
    \ChangeColorScheme{%
      fg=red!30!pink,%
      bg=green!10!black%
    }\end{verbatim}
  is used for this slide.
\end{frame}

\ChangeColorTheme{light}

\section{Examples}

\begin{frame}{Math equations}
  The Fourier transform
  \begin{equation}
    \hat{f}(\xi) = \int_{-\infty}^\infty
      f(x) e^{-2\pi ix\xi} \, dx .
  \end{equation}

  \pause

  The divergence theorem
  \begin{equation}
    \iiint_V (\vec{\nabla}\cdot\vec{F}) \, dV
      = \oiint_S (\vec{F}\cdot\vec{n}) \, dS .
  \end{equation}

  \pause

  The Dirac equation
  \begin{equation}
    \left( \beta mc^2 + c \Biggl( \sum_{n=1}^3 \alpha_n p_n \Biggr) \right)
      \psi(x, t) = i \hbar \frac{\partial \psi(x, t)}{\partial t} .
  \end{equation}
\end{frame}

\begin{frame}{Itemize}
  Here is an \texttt{itemize} example:
  \pause
  \begin{itemize}
    \item Item 1
    \pause
    \item Item 2
    \pause
    \item Item 3
  \end{itemize}
  \pause
  Of course, you can nest \texttt{itemize}.
\end{frame}

\appendix

\begin{frame}[fragile]{Making backup slides}
  It is often useful to have backup slides, which can be shown in case you need
  to give supplementary information. You can use the \verb|\appendix| command
  to make such backup slides.
\end{frame}

\end{document}
