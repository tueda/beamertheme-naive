\documentclass{beamer}
\usetheme{naive}

\title{\Naive}
\subtitle{Yet another simple beamer theme}
\author{Takahiro Ueda}
\institute{\url{https://github.com/tueda/beamertheme-naive}}
\date{30 March 2019}

% Well, look up the word "naiive" in the Urban Dictionary :-)
\newcommand{\Naive}{\texorpdfstring{\textsc{Na}ii{ve}}{Naive}}

\begin{document}

\maketitle

\section{Introduction}

\begin{frame}{Simple beamer theme}
  There are many beamer theme with simple and clean design.
  Here is another one!
\end{frame}

\section{Examples}

\begin{frame}{Math equations}
  The Fourier transform
  \begin{equation}
    \hat{f}(\xi) = \int_{-\infty}^\infty
      f(x) e^{-2\pi ix\xi} \, dx
  \end{equation}

  The divergence theorem
  \begin{equation}
    \iiint_V (\vec{\nabla}\cdot\vec{F}) \, dV
      = \oiint_S (\vec{F}\cdot\vec{n}) \, dS
  \end{equation}

  The Dirac equation
  \begin{equation}
    \left( \beta mc^2 + c \Biggl( \sum_{n=1}^3 \alpha_n p_n \Biggr) \right)
      \psi(x, t) = i \hbar \frac{\partial \psi(x, t)}{\partial t}
  \end{equation}
\end{frame}

\appendix

\begin{frame}[fragile]{Making backup slides}
  It is often useful to have backup slides, which can be shown in case you need
  to give supplementary information. You can use the \verb|\appendix| command
  to make such backup slides.
\end{frame}

\end{document}
